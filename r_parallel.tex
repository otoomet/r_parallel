% -*- mode: LaTeX; mode: flyspell; mode: Tex-Pdf -*-
\documentclass[pdftex]{beamer}
\DeclareGraphicsRule{&star}{mps}{&star}{}
\mode<presentation>
{
  \usetheme{Hannover}
  \setbeamercovered{transparent}
}
\usepackage[utf8]{inputenc}
\usepackage[T1]{fontenc}
\usepackage{array,xspace,icomma}
\usepackage{booktabs}
\usepackage{listings}
\usepackage{wasysym}
\input{isomath.tex}

\title[parallel R]{R in Parallel:\\
  From Laptop to Supercomputer}
\author{Ott Toomet}
\date{Seattle, Nov 14th, 2015}

\newcommand{\R}{\textsf{R}\xspace}
\lstset{language=R, extendedchars=true, stepnumber=2, breaklines=true,
  numberstyle=\scriptsize}

\begin{document}

\begin{frame}
  \titlepage
\end{frame}

\begin{frame}
  \frametitle{Outline}
  \tableofcontents
\end{frame}

\section{Background}

\begin{frame}
  \frametitle{\R}
  What is \R
  \begin{itemize}
  \item Tiobe top-20 programming language
  \item One of the most popular language for data analysis and statistics
  \item Superb graphics
    \pause
  \item[*] No built-in thread/parallel programming support
  \item[*] \code{parallel}-package for explicit (coarse) parallelism
    \begin{itemize}
    \item You explicitly call parallel code
    \end{itemize}
  \item[*] revolution R for implicit (fine) parallelism
    \begin{itemize}
    \item The software parallelizes standard constructions
      automatically
    \item Uses parallel libraries like \emph{ScaLAPACK}
    \end{itemize}
  \end{itemize}
\end{frame}

\begin{frame}
  \frametitle{Frontends}
  \begin{itemize}
  \item R-Studio
  \item ESS (Emacs)
  \item \code{R CMD Batch}
  \item \code{Rscript}
    \begin{itemize}
    \item Less bloated version of \code{R CMD BATCH}
    \item Does not load/save workspace (see below)
    \end{itemize}
  \end{itemize}
\end{frame}

\section{Clusters}
\subsection{Multicore}

\begin{frame}
  \frametitle{Multicore Parallelism}
  \begin{itemize}
  \item Almost all computers nowadays use multicore processors.
    \begin{itemize}
    \item Shared memory
    \item Fast
    \item Cheap
    \end{itemize}
  \item Let's use it!
  \item Example: 33 vs 75 seconds on my laptop (4 workers)
    \pause
  \item But it does not work on windows \frownie{}
  \end{itemize}
\end{frame}


\subsection{Socket}

\begin{frame}
  \frametitle{Socket Clusters}
  \begin{itemize}
  \item open new workers on different computers
    \begin{itemize}
    \item including on ``localhost''
    \end{itemize}
  \item Access these over internet
  \item Allows to use different computers
  \item Example: 50 vs 75 seconds on my laptop (2 workers)
  \item Example with 2 computers: 23 vs 75 seconds
    \pause
  \item Have to export data
  \item Communication slow
    \begin{itemize}
    \item \emph{top} shows the workers only partly (30\%) busy with
      small vectors
    \end{itemize}
  \end{itemize}
\end{frame}

\begin{frame}
  \frametitle{Conclusion So Far}
  Time in seconds
  \begin{itemize}
  \item length $10^{7} \Rightarrow$ grid $10\times10$
  \item length $10^{5} \Rightarrow$ grid $100\times100$
  \end{itemize}
  \begin{tabular}{l cccc}
    \toprule
    size      & single thread & multicore & socket (localhost) & 2 hosts\\
    \midrule
    $10^{7}$  & 75            & 33        & 35                 & 23\\
    $10^{5}$  & 74            & 33        & 121                & 30\\
    \bottomrule
  \end{tabular}
\end{frame}



\subsection{MPI}

\begin{frame}
  \frametitle{MPI Cluster}
  \begin{itemize}
  \item length $10^{7} \Rightarrow$ grid $10\times10$
  \item length $10^{5} \Rightarrow$ grid $100\times100$
  \end{itemize}
  \begin{tabular}{l cccc}
    \toprule
    size      & single thread & multicore & MPI & 2 hosts\\
    \midrule
    $10^{7}$  & 75            & 33        & 49                 & 23\\
    $10^{5}$  & 74            & 33        & 353                & 30\\
    \bottomrule
  \end{tabular}
\end{frame}

\section{Data Parallelism}

\begin{frame}
  \frametitle{Data Parallelism}
  \begin{itemize}
  \item Run the same code on different (chunks of) data
  \item \code{pbdMPI} library
  \item Works well with a cluster and \emph{mpirun}
  \item Can be used with distributed data (big data)
  \end{itemize}
\end{frame}


\begin{frame}
  \frametitle{Data Parallel}
  \begin{itemize}
  \item length $10^{7} \Rightarrow$ grid $10\times10$
  \item length $10^{5} \Rightarrow$ grid $100\times100$
  \item hyak:  $10^{7} \Rightarrow$ grid $100\times100$, 32 CPUs
  \end{itemize}
  \begin{tabular}{l cccc}
    \toprule
    size      & single thread & multicore & pbdMPI\\
    \midrule
    $10^{7}$  & 75            & 33        & 48     \\
    $10^{5}$  & 74            & 33        & 46    \\
    hyak      &               &           & 244 \\
    \bottomrule
  \end{tabular}
\end{frame}


\section{Indekseerimine}

\begin{frame}
  \frametitle{Indekseerimine}
  Kuidas vektori/maatriksi vajalikku elementi näperdada
  \begin{itemize}
  \item \code{length()} -- vektori pikkus
  \item \code{dim()} -- maatriksi mõõtmed
    \pause
  \item Erinevad indeksi tüübid:
    \begin{itemize}
    \item Täisarvud: \code{v[c(1,2,5)]}
    \item Negatiivsed täisarvud: \code{v[-1]}
    \item Loogiline indeks: \code{v[c(T, F, T)]}
    \item Komponentide nimed: \code{v[c("beta", "gamma")]}
    \item Kõik komponendid: \code{v[]}
    \end{itemize}
    \pause
  \item Vajalikele elementidele omistamine:\\
    \code{v[v < 0] <- 0}
  \item Andmebaasist selekteerimine:\\
    \code{data[data\$income > 0,]}
  \end{itemize}
\end{frame}



\section{Arvutamine}

\begin{frame}
  \frametitle{Arvutamine}
  \begin{itemize}
  \item Põhilised matemaatilised operatsioonid: \code{+}, $-$, $*$, $/$
  \item Loogikatehted \code{!}, \code{\&}, \code{|}, \code{==},
    \code{<}, \code{<=}, \code{\%in\%}, \dots
  \item Täisarvuline jagamine \textbackslash{}, jääk \code{\%\%}
  \item Maatrikskorrutis \code{\%*\%}
  \item Transponeerimine \code{t()}
    \pause
  \item Igasugu (vektor)funktsioonid: \code{log()}, \code{sqrt()},
    \code{exp()}, \dots
  \item Vektorite operatsioonid: \emph{recycling}
    \pause
  \item Numbriline optimeerimine: \code{nlm()}, \code{optim()}
  \item Numbriline võrrandite lahendamine: \code{uniroot()}
  \item Numbriline integraal: \code{area()}
  \item \code{options(digits=)}
  \end{itemize}
\end{frame}

\begin{frame}
  \frametitle{Näide}
  \begin{itemize}
  \item Laadi \code{http://www.obs.ee/\textasciitilde{}siim/ETU95.csv}
  \item Selekteeri vajalikud muutujad
  \item Arvuta vajalikud lähtesuurused
  \item Salvesta vahetulemused
  \end{itemize}
  \pause
  Väljavõte ETU1995 andmebaasist
  \begin{description}
  \item[C18E0000] bruttopalk 1994 sügisel, EEK
  \item[G21] haridus: 1,2 -- alg, 3,4 -- kesk; 5-7 -- kõrg
  \item[H01] perekonnaseis: 2,3 -- (vaba)abielu
  \item[I01EKOOD] elukoha kood: 1 -- Tallinn
  \item[J01] kas töötab uuringunädalal: 1 -- jah, 2 -- ei
  \item[L02A00] sünniaasta (kahekohaline)
  \item[L02D00] sugu: 1 -- mees, 2 -- naine
  \end{description}
\end{frame}

\section{Funktsioon}

\begin{frame}
  \frametitle{Funktsioonid}
  \begin{itemize}
  \item Interaktiivne käivitamine
  \item Argumendid
  \item Tulemused
    \pause
  \item Kontrollstruktuurid:
    \begin{itemize}
    \item \code{for()}
    \item \code{break}
    \item \code{if()}
    \item \code{else}
    \end{itemize}
    \pause
  \item Trükkimine: \code{cat()}
  \item Silumine: \code{browser()}, \code{traceback()}
  \item Meetodid
  \end{itemize}
\end{frame}



\section{Statistika}

\begin{frame}
  \frametitle{Statistilised mudelid}
  \begin{itemize}
  \item OLS -- \emph{linear model}
    \begin{align*}
      &\text{\texttt{> model <- lm(response \textasciitilde{} explanatory + variables)}}\\
      &\text{\texttt{> summary(model)}}
    \end{align*}
    \pause
  \item Logit/probit: osa üldistatud lineaarsetest mudelitest
    \emph{generalised linear models}:
    \begin{align*}
      &\text{\texttt{> model <- lm(response \textasciitilde{} explanatory + variables,}}\\
      &\text{\texttt{family=binomial(link="logit"))}}\\
      &\text{\texttt{> summary(model)}}
    \end{align*}
  \end{itemize}
\end{frame}

\begin{frame}
  \frametitle{Juhuslikud arvud}
  \begin{itemize}
  \item Jaotused:
    \begin{description}
    \item[\code{.unif}] ühtlane jaotus
    \item[\code{.norm}] normaaljaotus
    \item[\code{.exp}] eksponentjaotus
    \item[\code{.chisq}] $\chi^2$-jaotus
    \item[\code{.t}] $t$-jaotus
    \item[\code{.binom}] binoomjaotus
    \item[\code{.pois}] Poissoni jaotus
    \item \dots
    \end{description}
    \pause
  \item Statistilised tabelid:
    \begin{description}
    \item[\code{r\dots}] juhuslike arvude generaator (\code{rnorm})
    \item[\code{d\dots}] tõenäosustihedus (\code{dnorm})
    \item[\code{p\dots}] kumulatiivne jaotusfunktsioon (\code{pnorm})
    \item[\code{q\dots}] jaotuse kvantiilid (\code{qnorm})
    \end{description}
  \end{itemize}
\end{frame}

\begin{frame}
  \frametitle{Maximum likelihood}
  ML tuleb teha nagu alati:
  \begin{enumerate}
  \item Kirjuta likelihoodi funktsioon
  \item Maksimeeri parameetri järgi.
  \end{enumerate}
  Näide: genereerime normaaljaotusega juhuslikke arve ja leiame valimi
  keskmise:
  \begin{align}
    \loglik_i(\mu, \sigma; x_i) 
    &=
    \log \left(
      \frac{1}{\sqrt{2\pi}\sigma}
      \exp \left(
        - \frac{1}{2}
        \frac{(x_i - \mu)^2}{\sigma^2}
      \right)
    \right) =
    \\
    &=
    -\frac{1}{2} \log(2\pi)
    - \log \sigma
    - \frac{1}{2}
    \frac{(x_i - \mu)^2}{\sigma^2}
  \end{align}
\end{frame}



\section{Graafika}

\begin{frame}
  \frametitle{Graafika}
  \begin{itemize}
  \item Mõned näited
  \item Lihtne rida: \code{plot(x)}
  \item $x-y$ plot: \code{plot(x, y)}
  \item Histogramm: \code{hist(x)}
  \item Kernel tõenäosustihedus: \code{plot(density(x))}
  \item Võrdle jaotuse kvantiile: \code{qqnorm()}
  \item Funktsiooni kõver: \code{curve()}
  \end{itemize}
\end{frame}

\end{document}