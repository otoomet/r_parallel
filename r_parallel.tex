% -*- mode: LaTeX; mode: flyspell; mode: Tex-Pdf -*-
\documentclass[pdftex]{beamer}
\DeclareGraphicsRule{&star}{mps}{&star}{}
\mode<presentation>
{
  \usetheme{Hannover}
  \setbeamercovered{transparent}
}
\usepackage[utf8]{inputenc}
\usepackage[T1]{fontenc}
\usepackage{array,xspace,icomma}
\usepackage{booktabs}
\usepackage{listings}
\usepackage{wasysym}
\input{isomath.tex}

\title[parallel R]{R in Parallel:\\
  From Laptop to Supercomputer}
\author{Ott Toomet}
\date{Seattle, Nov 14th, 2015}

\newcommand{\R}{\textsf{R}\xspace}
\lstset{language=R, extendedchars=true, stepnumber=2, breaklines=true,
  numberstyle=\scriptsize}

\begin{document}

\begin{frame}
  \titlepage
\end{frame}

\begin{frame}
  \frametitle{Outline}
  \tableofcontents
\end{frame}

\begin{frame}
  \frametitle{Results Table}
  \begin{tabular}{l cccc}
    \toprule
    task      & single thread & multicore & socket (localhost) & ... \\
    \midrule
    $10^{7}$  & time in s     & ...       & ...                & ...\\
    $10^{5}$  & ...           & ...       & ...                & ...\\
    \bottomrule
  \end{tabular}
  
\end{frame}

\section{Background}

\begin{frame}
  \frametitle{\R}
  What is \R
  \begin{itemize}
  \item Tiobe top-20 programming language
  \item One of the most popular language for data analysis and statistics
  \item Superb graphics
    \pause
  \item[*] No built-in thread/parallel programming support
  \item[*] \code{parallel}-package for explicit (coarse) parallelism
    \begin{itemize}
    \item You explicitly call parallel code
    \end{itemize}
  \item[*] revolution R for implicit (fine) parallelism
    \begin{itemize}
    \item The software parallelizes standard constructions
      automatically
    \item Uses parallel libraries like \emph{ScaLAPACK}
    \end{itemize}
  \end{itemize}
\end{frame}

\begin{frame}
  \frametitle{Frontends}
  \begin{itemize}
  \item R-Studio
  \item ESS (Emacs)
  \item \code{R CMD Batch}
    \begin{itemize}
    \item loads/saves workspace
    \end{itemize}
  \item \code{Rscript}
    \begin{itemize}
    \item Less bloated version of \code{R CMD BATCH}
    \item Does not load/save workspace (see below)
    \end{itemize}
  \end{itemize}
\end{frame}

\section{Clusters}
\subsection{Multicore}

\begin{frame}
  \frametitle{Multicore Parallelism}
  \begin{itemize}
  \item Almost all computers nowadays use multicore processors.
    \begin{itemize}
    \item Shared memory
    \item Fast
    \item Cheap
    \end{itemize}
  \item \code{mclapply()}
  \item Let's use it!
  \item Example: 33 vs 75 seconds on my laptop (4 workers)
    \pause
  \item But it does not work on windows \frownie{}
  \item \code{browser()} does not work
  \item Not load balancing
  \end{itemize}
\end{frame}

\begin{frame}
  \frametitle{Example 1}
  \begin{itemize}
  \item Embarrasingly parallel task
  \item Compute normal density of a long vector
  \item Find maximum
  \item How many threads to run?
    \begin{itemize}
    \item \code{detectCores()}
    \end{itemize}
  \end{itemize}
\end{frame}

\begin{frame}[fragile]
  \frametitle{Single Core vs Multicore }
  Example on 8-core processor: 
\begin{lstlisting}
> x <- DGP(N)
> system.time(search1(nGrid=10))
Maximum -21427517 at mu = 0.5555556 and sigma = 2.222778 
   user  system elapsed 
158.041   2.590 160.717 
> system.time(search2(nGrid=10))
Maximum -21427517 at mu = 0.5555556 and sigma = 2.222778 
   user  system elapsed 
138.377   3.505  21.192 
\end{lstlisting}
  
\end{frame}

\subsection{Socket}

\begin{frame}
  \frametitle{Socket Clusters}
  \begin{itemize}
  \item open new workers on different computers
    \begin{itemize}
    \item including on ``localhost''
    \end{itemize}
  \item Access these over internet
  \item Allows to use multiple computers
  \item \code{makePSOCKcluster()}
  \item Example: 50 vs 75 seconds on my laptop (2 workers)
  \item Example with 2 computers: 23 vs 75 seconds
    \pause
  \item Have to export data
  \item Communication slow
    \begin{itemize}
    \item \emph{top} shows the workers only partly (30\%) busy with
      small vectors
    \end{itemize}
  \item Need password-less ssh connection
  \end{itemize}
\end{frame}

\begin{frame}
  \frametitle{Conclusion So Far}
  Time in seconds
  \begin{itemize}
  \item length $10^{7} \Rightarrow$ grid $10\times10$
  \item length $10^{5} \Rightarrow$ grid $100\times100$
  \end{itemize}
  \begin{tabular}{l cccc}
    \toprule
    size      & single thread & multicore & socket (localhost) & 2 hosts\\
    \midrule
    $10^{7}$  & 75            & 33        & 35                 & 23\\
    $10^{5}$  & 74            & 33        & 121                & 30\\
    \bottomrule
  \end{tabular}
\end{frame}



\subsection{MPI}

\begin{frame}
  \frametitle{MPI Cluster}
  \begin{itemize}
  \item length $10^{7} \Rightarrow$ grid $10\times10$
  \item length $10^{5} \Rightarrow$ grid $100\times100$
  \end{itemize}
  \begin{tabular}{l cccc}
    \toprule
    size      & single thread & multicore & MPI & 2 hosts\\
    \midrule
    $10^{7}$  & 75            & 33        & 49                 & 23\\
    $10^{5}$  & 74            & 33        & 353                & 30\\
    \bottomrule
  \end{tabular}
\end{frame}

\section{Data Parallelism}

\begin{frame}
  \frametitle{Data Parallelism}
  \begin{itemize}
  \item Run the same code on different (chunks of) data
  \item \code{pbdMPI} library
  \item Works well with a HPC and \emph{mpirun}
  \item Can be used with distributed data (big data)
  \end{itemize}
\end{frame}

\begin{frame}
  \frametitle{Example 1}
  Two processes execute the same code
  \begin{itemize}
  \item both generate different random numbers
  \item both print
  \end{itemize}
\end{frame}

\begin{frame}
  \frametitle{Example 2}
  Only master generates data
  \begin{itemize}
  \item Master shares data with all workers
  \end{itemize}
\end{frame}

\begin{frame}
  \frametitle{Example 3}
  Gridsearch example
  \begin{itemize}
  \item Master generates data
  \item Shares it to workers
  \item Workers calculate their share
  \item Master performs the final analysis
  \end{itemize}
\end{frame}

\begin{frame}
  \frametitle{Data Parallel}
  \begin{itemize}
  \item length $10^{7} \Rightarrow$ grid $10\times10$
  \item length $10^{5} \Rightarrow$ grid $100\times100$
  \item hyak:  $10^{7} \Rightarrow$ grid $100\times100$, 32 CPUs
  \end{itemize}
  \begin{tabular}{l cccc}
    \toprule
    size      & single thread & multicore & pbdMPI\\
    \midrule
    $10^{7}$  & 75            & 33        & 48     \\
    $10^{5}$  & 74            & 33        & 46    \\
    hyak      &               &           & 244 \\
    \bottomrule
  \end{tabular}
\end{frame}

\section[HPC]{High-Performance Cluster}

\begin{frame}
  \frametitle{High-Performance Cluster}
  UW hyak:
  \begin{itemize}
  \item 20,000 cpu cores
  \item 100TB memory
  \item MOAB cluster software
  \item TORQUE scheduler
  \item use \emph{pbs scripts}
    \begin{itemize}
    \item Tell the scheduler how much resources you want \dots
    \item \dots and run your stuff \smiley
    \end{itemize}
  \item submit the jobs by \emph{qsub}
  \end{itemize}
\end{frame}



\end{document}